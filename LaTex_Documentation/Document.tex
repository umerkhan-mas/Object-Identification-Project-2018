\documentclass{article}
\usepackage[T1]{fontenc}
\usepackage{amsmath}
\usepackage{hyperref}
\usepackage{geometry}
 \geometry{
 a4paper,
 total={170mm,257mm},
 left=20mm,
 top=20mm,
 }
\usepackage[utf8]{inputenc}
\title{Advanced Software Technology - Project 2018 - Week 1}
\author{Deepan Chakravarthi Padmanabhan\\ Muhammad Umer Ahmed Khan}
\date{December 2018}

\begin{document}

\maketitle
\newpage
\begin{center}
\title{Project 2018-Initial Requirement}
\end{center}
\begin{enumerate}
    \item Language of working:
    \begin{itemize}
        \item Java
    \end{itemize}
\item Coding standard:
\begin{itemize}
\item The Code Conventions for the Java Programming Language document updated on April 20, 1999. The doucment is available at: \href{https://www.oracle.com/technetwork/java/javase/documentation/codeconvtoc-136057.html}
{Oracle Java Code Convention}
\end {itemize}

\item Testing framework:
\begin{itemize}
\item JUnit Test 
\end {itemize}

\item{The data structure to communicate the object list to the code:}
\begin{itemize}
    \item Class, Arraylist - ListofObjects and the primitive data types.
\end{itemize}
\item The code structure along with the class/function type and definition:
\begin{itemize}
    \item Class name: Sensor 
    This class holds the entire input data as Arraylist of objects. For instance,\\
    
    [(knife,1, 99\%), (scissor, 2, 65\%), (spoon, 3, 33\%), (spoon, 4, 80\%), (keys, 5, 95\%)]
    [(knife,1, 55\%), (scissor, 2, 95\%), (fork, 3, 99\%), (spoon, 4, 99\%), (keys, 5, 95\%) ]\\
    
    Class name: SensorData
    This class holds the input for each modality as Arraylist of objects. For instance,\\
    
    [(knife,1, 99\%), (scissor, 2, 65\%), (spoon, 3, 33\%), (spoon, 4, 80\%), (keys, 5, 95\%)]\\

    Class name: SensorDataItem
    This class holds the data with respect to each object available in each modality image.
    The data is available as three variables of a class.For instance,\\
    
    (knife,1, 99\%)
    
\end{itemize}

\item Input and Output formats of data:
\begin{itemize}
    \item Input: Object of class Sensor
    \item Output: Object of class SensorDataItem
\end{itemize}

\item Test cases:
\begin{itemize}
    \item 
\end{itemize}


\item  Actual implementation:
\begin{itemize}
    \item .
\end{itemize}

\end{enumerate}


References:
\begin{enumerate}
    \item Coding convention, Available on: https://www.oracle.com/technetwork/java/javase/documentation/codeconvtoc-136057.html , Website: Oracle, Viewed on: 01.12.2018.
    \item Java Tutorials, Website: Tutorials point.
\end{enumerate}
\end{document}

